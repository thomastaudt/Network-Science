\documentclass{scrartcl}

%\usepackage{etex}
\usepackage[left=3cm,right=2.0cm,top=1.25cm,bottom=1.75cm,includehead,includefoot]
{geometry}
\usepackage{marginnote}
\reversemarginpar
%Math
\usepackage{amsfonts,amsmath,amssymb,amsthm, amstext}
%Language, encoding, ...
\usepackage[utf8]{inputenc}
\usepackage[T1]{fontenc}
\usepackage[english]{babel}
\usepackage{t1enc, lmodern, textcomp}
%Pictures
\usepackage{graphicx, color}
%\usepackage{latexsym}
%\usepackage{keyval}
%\usepackage{ifthen}
%\usepackage{moreverb}
\usepackage[shell]{gnuplottex}
%Colors
\usepackage[usenames,dvipsnames]{xcolor}
\definecolor{darkblue}{rgb}{0.0, 0.2, 0.6}
\renewcommand*{\marginfont}{\bf\color{darkblue}}
%Nice a/b fractions
\usepackage{xfrac}
%Make it possible to write (fat) upright greek letters
\usepackage{upgreek}

%For indices
\newcommand{\ix}[1]{_{\mathrm{#1}}}
\newcommand{\Ix}[1]{^{\mathrm{#1}}}

%Directly include svg images
%%%%%%%%%%%%%%%%%%%%%%%%%%%%%%%%%%%%%%%%%%%%%%%%%%
\newcommand{\executeiffilenewer}[3]{%
    \ifnum\pdfstrcmp{\pdffilemoddate{#1}}%
    {\pdffilemoddate{#2}}>0%
    {\immediate\write18{#3}}\fi%
}
\newcommand{\includesvg}[1]{%
    \executeiffilenewer{#1.svg}{#1.pdf}%
    {inkscape -z -D --file=#1.svg %
    --export-pdf=#1.pdf --export-latex}%
    \input{#1.pdf_tex}%
}
%%%%%%%%%%%%%%%%%%%%%%%%%%%%%%%%%%%%%%%%%%%%%%%%%%

%Make captions look okay
\usepackage[font=small,format=plain,labelfont=bf,up,up]{caption}

%Hyperlinks
\usepackage{hyperref}
\hypersetup{
            pdflang=en-EN,
            unicode=true,
            pdfauthor={Thomas Staudt, Erik Schultheis},
}

%Differential d's
\newcommand{\dif}{\mathrm{d}}
\newcommand{\tdif}[2]{\ensuremath{\frac{\dif#1}{\dif#2}}}
\newcommand{\pdif}[2]{\ensuremath{\frac{\partial#1}{\partial#2}}}
\newcommand{\ppdif}[2]{\ensuremath{\frac{\partial^{2}#1}{\partial#2^{2}}}}
%Degree
\newcommand{\degr}{^\circ}

\renewcommand{\refname} {Literature}
\renewcommand{\figurename}{\bf Figure}
\newcommand{\fs}[1]{\footnotesize #1}
%double slash
\newcommand{\git}{\mathbin{
  \mathchoice{\textbackslash\mkern-6mu\textbackslash}% \displaystyle
    {\textbackslash\mkern-6mu\textbackslash}% \textstyle
    {\textbackslash\mkern-5mu\textbackslash}% \scriptstyle
    {\textbackslash\mkern-5mu\textbackslash}}}% \scriptscriptstyle


%Bibliography
\usepackage[babel]{csquotes}
\usepackage[backend=bibtex8]{biblatex}


\newcommand*{\defeq}{\mathrel{\vcenter{\baselineskip0.5ex \lineskiplimit0pt
                     \hbox{\scriptsize.}\hbox{\scriptsize.}}}%
                     =}



%\bibliography{sources}




\begin{document}

\begin{titlepage}\centering
\textsc{\Large Institute For Nonlinear Dynamics \\[1.5ex] Universität Göttingen}

\vspace*{2cm}
{\huge A Practical Course On Network Science}
\vspace*{2cm}

\rule{\textwidth}{1pt}\\[0.5cm]
{\bfseries \huge Block D: \\[0.5cm] \huge \bfseries Additional Topics\\[0.5cm]}
\rule{\textwidth}{1pt}

\vspace*{4cm}

\begin{Large}\begin{tabular}{rl}
        \textbf{Participants:}  & Erik Schultheis                                \\    
                   & \textit{erik.schultheis@stud.uni-goettingen.de}\\[0.5cm]
                   & Thomas Staudt                                  \\
                   & \textit{thomas.staudt@stud.uni-goettingen.de}  \\[1.0cm]

       \textbf{Tutors:}        & Xiaozhu Zhang, Nora Molkenthin, Benjamin Schäfer, \\
                               & Malte Schröder                                    \\[1.0cm]
       \textbf{Deadline:}      & 28.07.2015
\end{tabular}\end{Large}

\vspace*{1.5cm}


\end{titlepage}

\tableofcontents
\clearpage

\section{Epidemics on Networks}
We consider an epidemic model in which each node is always in exactly one of the states \emph{susceptible}, \emph{infected} or \emph{recovered}. 
In every timestep, each infected node infects susceptible neighbours with probability $\beta$ and each infected node recovers with probability $\gamma$. 
The first five timesteps of evolving these dynamics on a BA network of $N=50$ nodes and $k=3$ are shown in figure \ref{}.

If all nodes connected to a susceptible node recover before they infect it, that node will never become infected (upper left node in figure \ref{}). How likely this is to happen depends on the infection and recovery probabilities $\beta$ and $\gamma$. 
It therefore seems sensible to define $q$ as the percentage of nodes that ever were infected. 
It can be calculated by simulating the network dynamics until there are no more infected nodes and is then given by the percentage of recovered nodes. 



\section{Inferring Network Structure}
The phrase \enquote{if you want to understand function, study structure}
uttered by Francis Crick is a common credo in biochemistry, applied
successfully to analyze and understand the function of large molecules. For
networks too, their structure, meaning the topology and the weigth
distributions, determine what the network can and can't do. This section's
task is to reverse this idea in a certain way: What can be deduced about
the structure of a network when having access to (some of) the output it
produces?


\subsection{Reconstructing Small Networks with Kuramoto Dynamics}
The approach used in order to reconstruct networks with oscillatory
Kuramoto dynamics can be found in the uploaded instruction and the
formulation of the task. The notation used in the following also coincides
with the notation of the instruction, if not introduced elsewise. The
pseudo inverse matrix was calculated with the function
\texttt{numpy.linalg.pinv()} provided by the \texttt{numpy} library.
The most important remarks regarding the simulation and the obtained
results are:
\begin{itemize}
    \item The integration time step for the Euler integration of the
        Kuramoto dynamic was $\dif t = 0.01$.
    \item Linearly spaced values $(T\ix{i})\ix{i=0}\Ix{M+2} \subset [0,
        T\ix{max}]$ were used as sample times. $M+2$ sample times had to be
        used in order to calculate the approximate derivatives for the border values. 
    \item The quality of the reconstructed adjacency matrix $\bar{K}$
        compared to the original matrix $K$ was measured by the quantity $Q_{95}$
        that was also used in a figure of the handout. It is defined
        in the publication of M. Timme and J. Casadiego, J. Phys. A: Math.
        Theor. 47 343001 (2014).
    \item The quality measures $\bar{Q_{95}}$ given below for one set $(N,
        M, T\ix{max})$ of parameters are the averaged $Q_{95}$ values for
        10000 trials.
\end{itemize}

\subsection{Reconstructing Larger Networks}
When trying to reconstruct larger networks, the applied method in the
considered parameter space seemes to get increasingly unsuitable. While for
$N=12$ and  $M = 100, T = 2$ the average quality was around
$\bar{Q}_{95}\approx 0.3$ it dropped to for $N=12$


% N = 5, T = 2, M = 100 -> Q_ = 0.94
% N = 5, T = 4, M = 150 -> Q_ = 0.96
% N = 5, T = 3, M = 50  -> Q_ = 0.92

% N = 10, 


\begin{figure}
    \centering
    \def\svgwidth{0.75\textwidth}
    \input{D211.pdf_tex}
    \caption{Reconstruction of oscillating networks with Kuramoto dynamics. The
        value $Q_{95}$ denotes the quality of the trial depicted in the
        respective column, while $\bar{Q}_{95}$ is the quality of 10000
        trials with the same $N, T$, and $M$ averaged.}
    \label{fig:211}
\end{figure}


%The dependencies of the reconstruction quality $Q_{95}$ on the parameters $N, M,$ and
%$T\ix{max}$ are presented in figures \ref{fig:??} to \ref{fig:??}.

%\paragraph{Dependency on $N$:}
%\paragraph{Dependency on $M$:}
%\paragraph{Dependency on $T\ix{max}$:}



\section{D4}

\end{document}

