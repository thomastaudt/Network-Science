%\documentclass[12pt,a4paper,titlepage,headinclude]{scrartcl}
\usepackage{etex}
\usepackage[left=3cm,right=2.0cm,top=2cm,bottom=2.5cm,includehead,includefoot]
{geometry}
%Matheunterstützung
\usepackage{amsfonts,amsmath,amssymb,amsthm, amstext}
%\usepackage[nointegrals]{wasysym}
%Sprach- und Schriftunterstützung, etc.
\usepackage[utf8]{inputenc}
\usepackage[T1]{fontenc}
\usepackage[english]{babel}
\usepackage{t1enc, lmodern, textcomp}
%\usepackage{uniinput}
%\usepackage{eurosym}
%Bildunterstützung
\usepackage{graphicx, color}
%\usepackage{gnuplottex}
%SI-Units
%\usepackage{booktabs}
%\usepackage{cancel}
%\usepackage{colortbl}
%\usepackage{helvet}
%\usepackage{mathpazo}
%\usepackage{multirow}
%\usepackage{listings}
%\usepackage{pgfplots}
\usepackage{xcolor}
\usepackage{icomma}
\usepackage{xfrac}
%\usepackage[per=frac, fraction=frac]{siunitx}
%\usepackage{pbox}
%\usepackage{enumitem}
%\usepackage{sectsty}
%\usepackage{setspace}
\usepackage{upgreek}

% Anpassen des Paragraph-Kopfes
%\usepackage{titlesec}
%\titleformat*{\paragraph}{\normalsize\bfseries}
%\newcommand{\myparagraph}[1]{\paragraph{#1}\mbox{}\\[0.1\baselineskip]}



% Damit man schneller die Schriftgröße variieren kann
\newcommand{\s}[1]{\small #1}
\newcommand{\f}[1]{\footnotesize #1}
\newcommand{\ix}[1]{_{\mathrm{#1}}}
\newcommand{\Ix}[1]{^{\mathrm{#1}}}

%%%%%%%%%%%%%%%%%%%%%%%%%%%%%%%%%%%%%%%%%%%%%%%%%%
%Direktes, aktuelles Einfügen von Inkscape-svgs:
\newcommand{\executeiffilenewer}[3]{%
    \ifnum\pdfstrcmp{\pdffilemoddate{#1}}%
    {\pdffilemoddate{#2}}>0%
    {\immediate\write18{#3}}\fi%
}
\newcommand{\includesvg}[1]{%
    \executeiffilenewer{#1.svg}{#1.pdf}%
    {inkscape -z -D --file=#1.svg %
    --export-pdf=#1.pdf --export-latex}%
    \input{#1.pdf_tex}%
}
%%%%%%%%%%%%%%%%%%%%%%%%%%%%%%%%%%%%%%%%%%%%%%%%%%

% Für mehrspaltige(getrennte) Tabellen:
\usepackage{multicol}

%Kopf- und Fußzeile
\usepackage{fancyhdr}
\pagestyle{fancy}
\fancyhf{}
\fancyheadoffset{0.2cm}
\fancyhead[RO,LE]{\small \nouppercase \leftmark}
%\fancyheadoffset{0.2cm}
\fancyfoot[C]{\thepage}

\fancypagestyle{myfancy}{
  \renewcommand\headrulewidth{0.0pt}
  \fancyhf{}
  \fancyhead[]{}
  \fancyfoot[C]{\thepage}
  \pagestyle{fancy}
}

%Captions für Abbildungen
\usepackage[font=small,format=plain,labelfont=bf,up,textfont=it,up]{caption}
%für die Aufnahme des Abb. und Lit-verzeichnisses in den ToC:
%%%\usepackage[nottoc, numbib]{tocbibind}
%%%\tocsection
% damit die Gleichungen stetig durchgezählt werden:
%%%\usepackage{chngcntr}
%%%\counterwithout{equation}{section}
% Hyperlinks für pdf
\usepackage{hyperref}
\hypersetup{
pdflang=en-EN,
unicode=true,
pdfauthor={Thomas Staudt},
}

\renewcommand{\vec}[1]{\boldsymbol{\mathbf{#1}}}
\newcommand{\hatvec}[1]{\hat{\boldsymbol{\mathbf{#1}}}}
\newcommand{\overbar}[1]{\mkern 1.5mu\overline{\mkern-1.5mu#1\mkern-1.5mu}\mkern 1.5mu}

%%%%%%%%%%%%%%%%%%%%%%%%%%%%%%%%%%%%%%%%%%
%Trennung

%%%%%%%%%%%%%%%%%%%%%%%%%%%
%Definitionen

%Times 10^n
\newcommand{\ee}[1]{\cdot 10^{#1}}
%Units
\newcommand{\unit}[1]{\,\mathrm{#1}}
%Differential d's
\newcommand{\dif}{\mathrm{d}}
\newcommand{\tdif}[2]{\ensuremath{\frac{\dif#1}{\dif#2}}}
\newcommand{\pdif}[2]{\ensuremath{\frac{\partial#1}{\partial#2}}}
\newcommand{\ppdif}[2]{\ensuremath{\frac{\partial^{2}#1}{\partial#2^{2}}}}
%Degree
\newcommand{\degr}{^\circ}
%Degree Celsius (C) symbol
\newcommand{\cel}{\,^\circ\mathrm{C}}
% Das qed Kästchen
%\newcommand{\qed}{\begin{flushright} $ ^\blacksquare $ \end{flushright} }
%Äquivalentzeichen
\newcommand{\hateq}{\ensuremath{\hat{=}}}
% Befehl für "Entspricht"-Zeichen
\newcommand{\corresponds}{\ensuremath{\mathrel{\widehat{=}}}}


\renewcommand{\refname} {Literature}
\renewcommand{\figurename}{\bf Fig.}


%%%%%%%%%%%%%%%%%%%%%%%%%%
%Hilfe

% Text umfließt Graphiken und Tabellen
% Beispiel:
% \begin{wrapfigure}[Zeilenanzahl]{"l" oder "r"}{breite}
%   \centering
%   \includegraphics[width=...]{grafik}
%   \caption{Beschriftung} 
%   \label{fig:grafik}
% \end{wrapfigure}
\usepackage{wrapfig}

% Mehrere Abbildungen nebeneinander
% Beispiel:
% \begin{figure}[htb]
%   \centering
%   \subfigure[Beschriftung 1\label{fig:label1}]
%   {\includegraphics[width=0.49\textwidth]{grafik1}}
%   \hfill
%   \subfigure[Beschriftung 2\label{fig:label2}]
%   {\includegraphics[width=0.49\textwidth]{grafik2}}
%   \caption{Beschriftung allgemein}
%   \label{fig:label-gesamt}
% \end{figure}
\usepackage{subfig}
\captionsetup[subtable]{position=top}



% Caption neben Abbildung
% Beispiel:
% \sidecaptionvpos{figure}{"c" oder "t" oder "b"}
% \begin{SCfigure}[rel. Breite (normalerweise = 1)][hbt]
%   \centering
%   \includegraphics[width=0.5\textwidth]{grafik.png}
%   \caption{Beschreibung}
%   \label{fig:}
% \end{SCfigure}
\usepackage{sidecap}


% für Bibliographie:
\usepackage[babel]{csquotes}
\usepackage[backend=bibtex8]{biblatex}

\bibliography{quellen}

%\sisetup{locale=EN}

