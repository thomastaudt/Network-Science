\documentclass{scrartcl}

%\usepackage{etex}
\usepackage[left=3cm,right=2.0cm,top=1.25cm,bottom=1.75cm,includehead,includefoot]
{geometry}
\usepackage{marginnote}
\reversemarginpar
%Math
\usepackage{amsfonts,amsmath,amssymb,amsthm, amstext}
%Language, encoding, ...
\usepackage[utf8]{inputenc}
\usepackage[T1]{fontenc}
\usepackage[english]{babel}
\usepackage{t1enc, lmodern, textcomp}
%Pictures
\usepackage{graphicx, color}
%\usepackage{latexsym}
%\usepackage{keyval}
%\usepackage{ifthen}
%\usepackage{moreverb}
\usepackage[shell]{gnuplottex}
%Colors
\usepackage[usenames,dvipsnames]{xcolor}
\definecolor{darkblue}{rgb}{0.0, 0.2, 0.6}
\renewcommand*{\marginfont}{\bf\color{darkblue}}
%Nice a/b fractions
\usepackage{xfrac}
%Make it possible to write (fat) upright greek letters
\usepackage{upgreek}

%For indices
\newcommand{\ix}[1]{_{\mathrm{#1}}}
\newcommand{\Ix}[1]{^{\mathrm{#1}}}

%Directly include svg images
%%%%%%%%%%%%%%%%%%%%%%%%%%%%%%%%%%%%%%%%%%%%%%%%%%
\newcommand{\executeiffilenewer}[3]{%
    \ifnum\pdfstrcmp{\pdffilemoddate{#1}}%
    {\pdffilemoddate{#2}}>0%
    {\immediate\write18{#3}}\fi%
}
\newcommand{\includesvg}[1]{%
    \executeiffilenewer{#1.svg}{#1.pdf}%
    {inkscape -z -D --file=#1.svg %
    --export-pdf=#1.pdf --export-latex}%
    \input{#1.pdf_tex}%
}
%%%%%%%%%%%%%%%%%%%%%%%%%%%%%%%%%%%%%%%%%%%%%%%%%%

%Make captions look okay
\usepackage[font=small,format=plain,labelfont=bf,up,up]{caption}

%Hyperlinks
\usepackage{hyperref}
\hypersetup{
            pdflang=en-EN,
            unicode=true,
            pdfauthor={Thomas Staudt, Erik Schultheis},
}

%Differential d's
\newcommand{\dif}{\mathrm{d}}
\newcommand{\tdif}[2]{\ensuremath{\frac{\dif#1}{\dif#2}}}
\newcommand{\pdif}[2]{\ensuremath{\frac{\partial#1}{\partial#2}}}
\newcommand{\ppdif}[2]{\ensuremath{\frac{\partial^{2}#1}{\partial#2^{2}}}}
%Degree
\newcommand{\degr}{^\circ}

\renewcommand{\refname} {Literature}
\renewcommand{\figurename}{\bf Figure}
\newcommand{\fs}[1]{\footnotesize #1}
%double slash
\newcommand{\git}{\mathbin{
  \mathchoice{\textbackslash\mkern-6mu\textbackslash}% \displaystyle
    {\textbackslash\mkern-6mu\textbackslash}% \textstyle
    {\textbackslash\mkern-5mu\textbackslash}% \scriptstyle
    {\textbackslash\mkern-5mu\textbackslash}}}% \scriptscriptstyle


%Bibliography
\usepackage[babel]{csquotes}
\usepackage[backend=bibtex8]{biblatex}


\newcommand*{\defeq}{\mathrel{\vcenter{\baselineskip0.5ex \lineskiplimit0pt
                     \hbox{\scriptsize.}\hbox{\scriptsize.}}}%
                     =}



%\bibliography{sources}




\newcommand*{\defeq}{\mathrel{\vcenter{\baselineskip0.5ex \lineskiplimit0pt
                     \hbox{\scriptsize.}\hbox{\scriptsize.}}}%
                     =}




\begin{document}

\begin{titlepage}\centering
\textsc{\Large Institute For Nonlinear Dynamics \\[1.5ex] Universität Göttingen}

\vspace*{2cm}
{\huge A Practical Course On Network Science}
\vspace*{2cm}

\rule{\textwidth}{1pt}\\[0.5cm]
{\bfseries \huge Block C: \\[0.5cm] \huge \bfseries ?\\[0.5cm]}
\rule{\textwidth}{1pt}

\vspace*{4cm}

\begin{Large}\begin{tabular}{rl}
        \textbf{Participants:}  & Erik Schultheis                                \\    
                   & \textit{erik.schultheis@stud.uni-goettingen.de}\\[0.5cm]
                   & Thomas Staudt                                  \\
                   & \textit{thomas.staudt@stud.uni-goettingen.de}  \\[1.0cm]

       \textbf{Tutors:}        & Dr. Nora Molkenthin, Benjamin Schäfer, Malte Schröder, ?  \\[1.0cm]
       \textbf{Deadline:}      & ?
\end{tabular}\end{Large}

\vspace*{1.5cm}


\end{titlepage}

\tableofcontents
\clearpage

\section{whatever C1 was}

\section{Network Motifs}
Until now we have looked at network structure only from a global perspective (e.g by considering degree histograms, avg. shortest path length etc.), but for real network applications much information can probabily be found in more local structures. Therefore, we introduce the notion of a network motif. This is a connected subgraph that occurs often enough in the given graph that we can consider it relevant. One possible criterion for relevance is the comparison of occurances in the investigated graph compared to a random graph that shares the global characteristics of the former, i.e. its nodes have the same degrees.

We define the relevance $r$ of a $k$-motif $M$ of a graph $G$ as the precentage of $k$-node subgraphs $S_k$ of $G$ that are isomorphic to $M$, i.e. where there exists a perutation of its nodes so that it is identical to $M$. Let $\chi_M$ be that characteristic function of $M$, i.e. $\chi_M(S) = 1$ iff $S$ is isomorphic to $M$ [WELL, TO ONE REPRESENTANT OF M]. We can thus write $r$ as (for finite graphs)
\begin{align}
 r = (\# S_k)^{-1} \cdot \sum_{g \in S_k} \chi_M (g).
\end{align}
Since the number of possible $k-$node subgraphs grows exponentially fast, evaluating this sum directly is not feasible for large graphs. Therefore, we sample this sum (i.e. Monte-Carlo technique), where $p(g)$ denotes the probability of chosing the sample $g$:
\begin{align}
 r \approx \frac{1}{N} \sum_{i=1}^{N} \frac{\chi_M (g_i)}{p(g_i)}.
\end{align}

We now focus our attention on $3$-motifs. Since $M$ is connected, we can restrict sampling to connected subgraphs in the following way: \textbf{1)} choose a random starting node $A$ \textbf{2)} choose $B$ as a neighbour of $A$ \textbf{3)} choose $C$ as a common neighbour of $A$ and $B$. The probability of taking this sample is ($N(k)$ denoting neighbours of $k$)
\begin{align}
 p(A,B,C) = \underbrace{1/N}_{A} \cdot \underbrace{1/\#N(A)}_{B} \cdot \underbrace{1/\#(N(A) \cap N(B))}_{C}.
\end{align}
[TODO do we need to consider permutations here? in the end, this just affects the normalisation factor and reative weights stay intact, but it would be nice to have a self-contained argument.]

If we assume that the graph in question is large enough, and that there is no correlation between individual node degrees and the motifs, the different values of $p(A,B,C)$ will average out, and we can disregard the dependence to sampling probability [TODO again, this affects normalization, right?]. 

We assume this probability to hold for random directed Erdös-Renyi and Watts-Strogatz graphs [This is unrealistic, at least for WS], for which we investigate the occurance of 3-motifs.

Since $p=4/99$ is low in the Erdös-Renyi network, we get sparse connectivity, and thus motifs with less edges are more likely. Since there are two possible realisations for motiv $2$ compared to $1$ and $4$, $2$ has twice the probability, which is $\SI{45}{\percent}$. For motif $5$ another connection is required, so it is less probable by a factor $p/(1-p)*3 = 0.12$ (the $3$ results from the $6$ total possible edge direction distributions that result in this pattern, compared to the two for pattern $2$), which gives $r_5=\SI{5.4}{\percent}$ in accordance with the numerical findings. Pattern $11$ contains so another edge, so it is again less probable by a factor of order $0.1$.

Since Watts-Strogatz starts off with a ring structure that consists entirely of the motifs 13 (neighbouring points) and 8 (if C is not a direct neighbour of A and B), these patterns, and slight variations (i.e. 12, 7 and 3) comprise the majority of motifs encountered in those graphs. We find motif 2 with $\SI{5}{\percent}$ probability, $5$ with $\SI{0}{\percent}$ and 11 with $\SI{1}{\percent}$, which means that those structures are not significant motifs of Small-World networks compared to random Erdös-Renyi networks.

\begin{table}
	\centering
		\begin{tabular}{c|*{13}{l}}
		\toprule
		Topology & \num{1} & 2 & 3 & 4 & 5 &6 & 7 & 8 & 9 & 10 &11 & 12& 13  \\ \midrule
		Watts Strogatz & \SI{2}{\percent} & \SI{5}{\percent} & \SI{15}{\percent} & \SI{2}{\percent} & \SI{0}{\percent} & \SI{1}{\percent} & \SI{15}{\percent} & \SI{25}{\percent} & \SI{0}{\percent} & \SI{1}{\percent} & \SI{1}{\percent} & \SI{12}{\percent} & \SI{21}{\percent} \\
		Erdös Renyi & \SI{22}{\percent} & \SI{45}{\percent} & \SI{2}{\percent} & \SI{22}{\percent} & \SI{5}{\percent} & \SI{0}{\percent} & \SI{2}{\percent} & \SI{0}{\percent} & \SI{2}{\percent} & \SI{0}{\percent} & \SI{0}{\percent} & \SI{0}{\percent} & \SI{0}{\percent}\\
		\bottomrule
		\end{tabular}
	\caption{Frequency of occurance of motif in directed Watts-Strogatz and Erdös-Renyi networks. Motif numbers are taken from the presentation handout. The results were obtained by taking 10000 samples in 100 different, randomly generated networks of the designated topology. Each graph contained 100 nodes and the average degree was $k_{\mathrm{in}} = k_{\mathrm{out}}=4$, the rewire probability for Watts-Strogatz was chosen to be $\SI{10}{\percent}$.}
	\label{tab:motifs}
\end{table}

\paragraph{Motifs in real networks}
A network describing hierarchical dependencies (e.g. predator-prey networks) will mostly contain the motifs 1 or 4, depending on the direction in which the relationsships are defined, as well as 2. Any motif containing bidirectional edges (3, 6-8, 10-13) or loops (9) will not appear. There might be situations in which structure 5 appears, e.g. omnivores that eat both plants and the animals that eat those.

A road network, on the other hand, will contain mostly bidirectional edges and therefore show mostly motifs 8 and 13. 

\section{Optimal Network Design}
For a given set of evenly and randomly spaced points on the unit square, we want to determine the set of edges that constitute an optimal (in the sense of optimizing both construction/maintenance cost as well as minimizing travel times) transportation network.

We approximate the maintenance cost of the network as proportional to the total edge length. For travel times, we take into account both the actual time en route as well as possible mandatory delays at each stop, weighted against each other with a facotr $\delta$. The total network cost function is then a weighted sum of both terms, with a weight of $\alpha$ to relate them.  
 

\end{document}
