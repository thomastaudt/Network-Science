\documentclass{scrartcl}

%\usepackage{etex}
\usepackage[left=3cm,right=2.0cm,top=1.25cm,bottom=1.75cm,includehead,includefoot]
{geometry}
\usepackage{marginnote}
\reversemarginpar
%Math
\usepackage{amsfonts,amsmath,amssymb,amsthm, amstext}
%Language, encoding, ...
\usepackage[utf8]{inputenc}
\usepackage[T1]{fontenc}
\usepackage[english]{babel}
\usepackage{t1enc, lmodern, textcomp}
%Pictures
\usepackage{graphicx, color}
%\usepackage{latexsym}
%\usepackage{keyval}
%\usepackage{ifthen}
%\usepackage{moreverb}
\usepackage[shell]{gnuplottex}
%Colors
\usepackage[usenames,dvipsnames]{xcolor}
\definecolor{darkblue}{rgb}{0.0, 0.2, 0.6}
\renewcommand*{\marginfont}{\bf\color{darkblue}}
%Nice a/b fractions
\usepackage{xfrac}
%Make it possible to write (fat) upright greek letters
\usepackage{upgreek}

%For indices
\newcommand{\ix}[1]{_{\mathrm{#1}}}
\newcommand{\Ix}[1]{^{\mathrm{#1}}}

%Directly include svg images
%%%%%%%%%%%%%%%%%%%%%%%%%%%%%%%%%%%%%%%%%%%%%%%%%%
\newcommand{\executeiffilenewer}[3]{%
    \ifnum\pdfstrcmp{\pdffilemoddate{#1}}%
    {\pdffilemoddate{#2}}>0%
    {\immediate\write18{#3}}\fi%
}
\newcommand{\includesvg}[1]{%
    \executeiffilenewer{#1.svg}{#1.pdf}%
    {inkscape -z -D --file=#1.svg %
    --export-pdf=#1.pdf --export-latex}%
    \input{#1.pdf_tex}%
}
%%%%%%%%%%%%%%%%%%%%%%%%%%%%%%%%%%%%%%%%%%%%%%%%%%

%Make captions look okay
\usepackage[font=small,format=plain,labelfont=bf,up,up]{caption}

%Hyperlinks
\usepackage{hyperref}
\hypersetup{
            pdflang=en-EN,
            unicode=true,
            pdfauthor={Thomas Staudt, Erik Schultheis},
}

%Differential d's
\newcommand{\dif}{\mathrm{d}}
\newcommand{\tdif}[2]{\ensuremath{\frac{\dif#1}{\dif#2}}}
\newcommand{\pdif}[2]{\ensuremath{\frac{\partial#1}{\partial#2}}}
\newcommand{\ppdif}[2]{\ensuremath{\frac{\partial^{2}#1}{\partial#2^{2}}}}
%Degree
\newcommand{\degr}{^\circ}

\renewcommand{\refname} {Literature}
\renewcommand{\figurename}{\bf Figure}
\newcommand{\fs}[1]{\footnotesize #1}
%double slash
\newcommand{\git}{\mathbin{
  \mathchoice{\textbackslash\mkern-6mu\textbackslash}% \displaystyle
    {\textbackslash\mkern-6mu\textbackslash}% \textstyle
    {\textbackslash\mkern-5mu\textbackslash}% \scriptstyle
    {\textbackslash\mkern-5mu\textbackslash}}}% \scriptscriptstyle


%Bibliography
\usepackage[babel]{csquotes}
\usepackage[backend=bibtex8]{biblatex}


\newcommand*{\defeq}{\mathrel{\vcenter{\baselineskip0.5ex \lineskiplimit0pt
                     \hbox{\scriptsize.}\hbox{\scriptsize.}}}%
                     =}



%\bibliography{sources}




\begin{document}

\begin{titlepage}\centering
\textsc{\Large Institute For Nonlinear Dynamics \\[1.5ex] Universität Göttingen}

\vspace*{2cm}
{\huge A Practical Course On Network Science}
\vspace*{2cm}

\rule{\textwidth}{1pt}\\[0.5cm]
{\bfseries \huge Block C: \\[0.5cm] \huge \bfseries Community and Structure\\[0.5cm]}
\rule{\textwidth}{1pt}

\vspace*{4cm}

\begin{Large}\begin{tabular}{rl}
        \textbf{Participants:}  & Erik Schultheis                                \\    
                   & \textit{erik.schultheis@stud.uni-goettingen.de}\\[0.5cm]
                   & Thomas Staudt                                  \\
                   & \textit{thomas.staudt@stud.uni-goettingen.de}  \\[1.0cm]

       \textbf{Tutors:}        & Xiaozhu Zhang, Nora Molkenthin, Benjamin Schäfer, Malte Schröder  \\[1.0cm]
       \textbf{Deadline:}      & 01.07.2015
\end{tabular}\end{Large}

\vspace*{1.5cm}


\end{titlepage}

\tableofcontents
\clearpage
\section{Communities in Networks}
When searching for (local) properties and ways of classifying networks, one
interesting aspect is the \emph{community structure} of the network in
question. That is, how can the network reasonably be divided in connected
components so that the nodes of the same component interact stronger than
the nodes of different components do (meaning that the components form
\enquote{communities}).

One way of obtaining the community structure of a network $G = (V,E)$ is by
disintegrating it (via removing edges) according to the edges' values of
the \emph{shortest path betweenness centrality} $g$. The betweenness
centrality $g(e)$ of an edge $e$ measures how \enquote{central} an edge is
for efficient connectivity in the network. More precisely it describes how
often $e$ is averagely used when taking the shortest path between two
nodes, thus
\begin{equation}\label{eq:betweenness}
    g(e) = \sum_{s,t\in V} \frac{\sigma(s,t | e)}{\sigma(s,t)}~,
\end{equation}
where $\sigma(s,t)$ is the number of shortest paths between the nodes $s$
and $t$ and $\sigma(s,t |e)$ is the number of those paths that go through
$e$. So edges with a high value of $g$ are likely to be important for
mediating between different communities of the graph.

Having obtained a breakdown of the network in connected subgraphs
(understood as communities) one must next be able to assess the quality and
strength of this community structure, e.g.\ in order to find an optimal
number of communities. This can be done by calculating the
\emph{modularity} $Q(H)$ for a partitioning $H$ of the network $G$ (the
formula used to calculate $Q$ can be found on the respective handout for
C.1.1).
In the following, the variable $h$ will stand for the number of communities
for a given partitioning $H$ of a graph $G$.

Another property used to characterize given community structures is to
calculate the stability $r(H, t)$ of the given division $H$ of $G$ in
partitions for different time scales $t$. This measure tells us, how likely
it is to end up in the same community $c\in H$ that we started in if
a random walk of length $t\in\mathbb{N}$ was carried out.Though the
stability could approximately be calculated by simulating random walks on
the graph, the analytic approach of the handout for C.1.2 was used for the
calculations below.


\subsection{The Karate Club Network}
The first network $G_0$ to be analyzed for its group structure was a given
network for a karate club. The original graph as well as the achieved
decompositions in communities for $h=2$ and $h=5$ are visualized in
\ref{fig:12_gr}.
Next the modularity of each community structure obtained for the different
numbers $h$ of communities was calculated. This modularity curve is
depicted in figure \ref{fig:12_mod} and shows that the found partitionings
with 2 and 5 communities are in fact the ones with the highest modularity
$Q$.

\subsection{Artificial Networks With Strong Community Structure}
Besides the karate network $G_0$ the same algorithms for community
detection and their modularity were used for other networks as well.
They are visualized -- together with selected decompositions -- in figure
\ref{fig:12_gr}.
\begin{itemize}
    \item [\textbf{$G_1$}] A simple deterministic network consisting of
        a triangle of center-nodes each connected to hexagons.
    \item [\textbf{$G_2$}] A more complex deterministic network consisting
        of 9 center-nodes each connected to 20-gons.
    \item [\textbf{$G_3$}] A random network with a grouping structure of
        $4\times 25$ nodes. Each node is connected to nodes in its own
        group with $p_g=0.75$ and to nodes in other groups with $p_o
        = 0.1$.
\end{itemize}

For each of these artificially community-structured networks the
algorithm based on the shortest-path betweenness centrality managed to find
the obvious community structures (as is depicted in figure
\ref{fig:12_gr}) and these obvious communities were indeed global
maxima of the modularity $Q$ when compared to the other communities
obtained (see figure \ref{fig:12_mod}).

As a side node, however, the algorithm failed to obtain the community
structure for $G_3$ if the difference between $p_g$ and $p_i$ was chosen
to be smaller. For $p_g = 0.65$ and $p_i = 0.15$ e.g.\ the desired
decomposition in four rings was not achieved.

\subsection{From Random Walks to Community}
As the last task of this section, we calculated the stability curves
$t\mapsto r(H, t)$ of the four networks $G_0$, $G_1$, $G_2$, and $G_3$ for
different partitionings $H$ (obtained in the same way as above). The
resulting stability curves for time scales $1\le t \le 100$ can be seen in
figure \ref{fig:13}.

The results are as expected and show that the stability is in most cases
indeed best for the same communities that also exhibit the highest
modularity. For large time scales though, one can see for $G_2$ and $G_3$ that
partitionings with fewer communities than the expected value $h = 9$ and
$h=4$ are more stable. Since longer random walks cover longer distances on
this behaviour is not surprising.

\begin{figure}[bcht]
    \centering
    \gnuplotloadfile[terminal=epslatex, terminaloptions={color size 6.5,4.0}]{pictures/11.gp}
    \caption{Modularity curves of the networks $G_0$ (karate), $G_1, G_2,$
        and $G_3$ described in the text. For community structures
        corresponding to the vertical lines explicit visualizations may be
        found in figure \ref{fig:12_gr}.}
    \label{fig:12_mod}
\end{figure}


\begin{figure}
    \centering
    \def\svgwidth{0.8\textwidth}
    %\input{pictures/12.pdf_tex}
    \input{12.pdf_tex}
    \caption{Visualizations of the graphs $G_0$, $G_1$, $G_2$, and $G_3$
    before ($h = 1$) and after ($h > 1$) the algorithm for finding
    communities -- using the betweenness centrality -- has been applied,
    where $h$ denotes the number of communities in the respective
    partitioning $H$. For $G_0$ the colors were chosen based on the
    knowledge of the partitioning for $h=2$.  For $G_1$, $G_2$, and $G_3$
    the colors were chosen according to the obvious community
    structur we had in mind when creating these networks. All of these
    obvious structures were found by the algorithm: $h = 3$ for $G_1$,
    $h=9$ for $G_2$, $h=4$ for $G_4$.}
    \label{fig:12_gr}
\end{figure}


\begin{figure}
    \centering
    \gnuplotloadfile[terminal=epslatex, terminaloptions={color size 6.35,8.0}]{pictures/13.gp}
    \caption{Double-logarithmic plots of the stability curves for selected
    partitionings $H$ of the networks $G_0$ (karate), $G_1, G_2,$ and
    $G_3$ described in the text. The numbers $h_\mathrm{max}$ printed above
    the graphs correspond to the number of communities for which the
    stability was highest in the respective time scale interval.}
    \label{fig:13}
\end{figure}


\end{document}
