\documentclass{scrartcl}

%\usepackage{etex}
\usepackage[left=3cm,right=2.0cm,top=1.25cm,bottom=1.75cm,includehead,includefoot]
{geometry}
\usepackage{marginnote}
\reversemarginpar
%Math
\usepackage{amsfonts,amsmath,amssymb,amsthm, amstext}
%Language, encoding, ...
\usepackage[utf8]{inputenc}
\usepackage[T1]{fontenc}
\usepackage[english]{babel}
\usepackage{t1enc, lmodern, textcomp}
%Pictures
\usepackage{graphicx, color}
%\usepackage{latexsym}
%\usepackage{keyval}
%\usepackage{ifthen}
%\usepackage{moreverb}
\usepackage[shell]{gnuplottex}
%Colors
\usepackage[usenames,dvipsnames]{xcolor}
\definecolor{darkblue}{rgb}{0.0, 0.2, 0.6}
\renewcommand*{\marginfont}{\bf\color{darkblue}}
%Nice a/b fractions
\usepackage{xfrac}
%Make it possible to write (fat) upright greek letters
\usepackage{upgreek}

%For indices
\newcommand{\ix}[1]{_{\mathrm{#1}}}
\newcommand{\Ix}[1]{^{\mathrm{#1}}}

%Directly include svg images
%%%%%%%%%%%%%%%%%%%%%%%%%%%%%%%%%%%%%%%%%%%%%%%%%%
\newcommand{\executeiffilenewer}[3]{%
    \ifnum\pdfstrcmp{\pdffilemoddate{#1}}%
    {\pdffilemoddate{#2}}>0%
    {\immediate\write18{#3}}\fi%
}
\newcommand{\includesvg}[1]{%
    \executeiffilenewer{#1.svg}{#1.pdf}%
    {inkscape -z -D --file=#1.svg %
    --export-pdf=#1.pdf --export-latex}%
    \input{#1.pdf_tex}%
}
%%%%%%%%%%%%%%%%%%%%%%%%%%%%%%%%%%%%%%%%%%%%%%%%%%

%Make captions look okay
\usepackage[font=small,format=plain,labelfont=bf,up,up]{caption}

%Hyperlinks
\usepackage{hyperref}
\hypersetup{
            pdflang=en-EN,
            unicode=true,
            pdfauthor={Thomas Staudt, Erik Schultheis},
}

%Differential d's
\newcommand{\dif}{\mathrm{d}}
\newcommand{\tdif}[2]{\ensuremath{\frac{\dif#1}{\dif#2}}}
\newcommand{\pdif}[2]{\ensuremath{\frac{\partial#1}{\partial#2}}}
\newcommand{\ppdif}[2]{\ensuremath{\frac{\partial^{2}#1}{\partial#2^{2}}}}
%Degree
\newcommand{\degr}{^\circ}

\renewcommand{\refname} {Literature}
\renewcommand{\figurename}{\bf Figure}
\newcommand{\fs}[1]{\footnotesize #1}
%double slash
\newcommand{\git}{\mathbin{
  \mathchoice{\textbackslash\mkern-6mu\textbackslash}% \displaystyle
    {\textbackslash\mkern-6mu\textbackslash}% \textstyle
    {\textbackslash\mkern-5mu\textbackslash}% \scriptstyle
    {\textbackslash\mkern-5mu\textbackslash}}}% \scriptscriptstyle


%Bibliography
\usepackage[babel]{csquotes}
\usepackage[backend=bibtex8]{biblatex}


\newcommand*{\defeq}{\mathrel{\vcenter{\baselineskip0.5ex \lineskiplimit0pt
                     \hbox{\scriptsize.}\hbox{\scriptsize.}}}%
                     =}



%\bibliography{sources}




\newcommand*{\defeq}{\mathrel{\vcenter{\baselineskip0.5ex \lineskiplimit0pt
                     \hbox{\scriptsize.}\hbox{\scriptsize.}}}%
                     =}




\begin{document}

\begin{titlepage}\centering
\textsc{\Large Institute For Nonlinear Dynamics \\[1.5ex] Universität Göttingen}

\vspace*{2cm}
{\huge A Practical Course On Network Science}
\vspace*{2cm}

\rule{\textwidth}{1pt}\\[0.5cm]
{\bfseries \huge Block B: \\[0.5cm] \huge \bfseries Oscillating Networks [?]\\[0.5cm]}
\rule{\textwidth}{1pt}

\vspace*{4cm}

\begin{Large}\begin{tabular}{rl}
        \textbf{Participants:}  & Erik Schultheis                                \\    
                   & \textit{erik.schultheis@stud.uni-goettingen.de}\\[0.5cm]
                   & Thomas Staudt                                  \\
                   & \textit{thomas.staudt@stud.uni-goettingen.de}  \\[1.0cm]

       \textbf{Tutors:}        & Dr. Nora Molkenthin, Benjamin Schäfer, Malte Schröder  \\[1.0cm]
       \textbf{Deadline:}      & 09.05.2015
\end{tabular}\end{Large}

\vspace*{1.5cm}


\end{titlepage}

\tableofcontents
\clearpage

\section{Synchronization in Complex Networks}
\subsection{Kuramoto Oscillators}
The \emph{Kuramoto model} describes the phase coupling of a network of
oscillators. A network with $\mathrm{N}$ oscillators that have phases
$(\phi\ix{i}(t))\ix{i=1}\Ix{N}$ at time $t$ and which possess preferential
frequencies $(\omega\ix{i})\ix{i=1}\Ix{N}$ is coupled in the Kuramoto model via the
following system of differential equations:
\begin{equation}
    \label{eq:kuramoto}
    \dot{\phi}\ix{i}(t) = \omega\ix{i} - \sigma~\sum\ix{j=1}\Ix{N} a\ix{ij} \sin\big(\phi\ix{i}(t) - \phi\ix{j}(t)\big)~.
\end{equation}
Here $\sigma$ describes the \emph{coupling strength} and
$(a\ix{ij})\ix{i,j=1}\Ix{N}$ is the adjacency matrix of the unweighted graph of
the network, meaning that $a\ix{ij} \in \{0, 1\}$ for all $1\le
\mathrm{i},\mathrm{j}\le \mathrm{N}$. Besides $\sigma$ the variable $K$
is sometimes used to characterize the coupling strength, where one has the
relation
\begin{equation}
    \sigma = \frac{K}{\mathrm{\langle k \rangle}}
\end{equation}
with the average degree $\langle k \rangle$ of the network.
A possible measure for the synchronization of the oscillators is given by
\begin{equation}
    r(t) = \frac{1}{N}~\left| \sum\ix{j=0}^N \mathrm{e}^{i~\phi\ix{j}(t)} \right|~,
\end{equation}
which is near $0$ for unsynchronized networks and near $1$ for synchronized
ones. The quantity $r$ (averaged over some period of time at the end of
a long simulation) understood as function of the coupling strength
$K$ for a fixed network $a\ix{ij}$ can be used to examine percolation, where
the system suddenly changes from an unsynchronized to a strongly
synchronized state. The critical percolation value $\sigma_c$ for this process is named
the \emph{critical coupling strength}.


\subsection{Critical Coupling for Different Kuramoto Oscillators}
For all of the following simulation results, networks of size $\mathrm{N}=100$ were used.
Single values of $r$ were calculated by numerically solving equation
\eqref{eq:kuramoto} with $\dif t=0.01$ from $t\ix{start} = 0$ up to
$t\ix{end} = 10$ by a simple Euler algorithm, using
values for $\omega_i$ drawn from a standard normal distribution $g \sim
\mathcal{N}(0, 1)$. Then $r$ was taken to be the mean of $r(t)$ over the
last $100$ time steps. 
In order to compare the obtained results with analytic predictions, the
following mean-field approximation was used to calculate $\sigma_c$:
\begin{equation}\label{eq:s_crit}
    \sigma_c = \frac{2}{\pi\,g(0)}~\frac{\langle k \rangle}{\langle k^2 \rangle} = 2\,\sqrt{\frac{2}{\pi}}~\frac{\langle k \rangle}{\langle k^2 \rangle}~.
\end{equation}
Here the second equality holds because of $g \sim \mathcal{N}(0, 1)$.

\paragraph{Completely Connected Networks}
The first simulations were carried out for completely connected networks,
making the coupling \enquote{all-to-all}. In this case, $\omega\ix{ij} = 1$
for all $1\le \mathrm{i},\mathrm{j}\le \mathrm{N}$ and $\langle k \rangle
= \mathrm{N}$ as well as $\langle k^2 \rangle = \mathrm{N}^2$. The results obtained are
presented in figure \ref{fig:B11}. One can see that the analytically
predicted value $\sigma_c = 2/\big(\pi\,g(0)\,\mathrm{N}\big)\approx 0.016$ of
the critical coupling strength is consistent with the results obtained.

\paragraph{Erdös-Renyi Networks}
Next the coupling behaviour of (undirected) Erdös-Renyi networks was
analyzed for three different values of the parameter $p$: $p\in\{0.1, 0.3, 0.5\}$.
Here the predicted values for $\sigma_c$ were calculated numerically by
averaging formula \eqref{eq:s_crit} over 30 trial networks each and are
included in figure \ref{fig:B12} a) together with the simulation results.

One again discerns that the analytic predictions work quite well for
describing the percolation process.  The reason that the needed coupling
strength for synchronization is higher than for the fully connected network
is of course that fewer connections cause weaker mutual influences of the
single oscillators.

\paragraph{Scale Free Networks}
The last network type examined for synchronizing were scale free networks,
implemented by using Barab\'asi-Albert networks with numbers $m\in\{5, 10,
15\}$ of edges added in each step. The critical values were again
calculated numerically as stated in the last paragraph.  The results can be
seen in figure \ref{fig:B12} b), where the analytic values once more seem
to describe the percolation appropriately.

(MISSING: Why are the values the way they are??)

\subsection{Synchronization Conditions}
There are several conditions that are necessary for the network to be able
to synchronize. A very simple requirement is that the network's graph must
be \emph{connected}, since different connected components can't influence the phasing
of one another at all (depicted in figure \ref{fig:B13} a). In general of
course, the stronger the interconnectivity of the oscillators, the better
the synchronization works. 

Another way to destroy the ability to synchronize (besides taking more than
one connected component) is to introduce directed edges and provide some
restrictions for allowed edges. For example, a central component $C$ could
be connected to two other components $A$ and $B$ only by arrows from $A$
and $B$ pointing to $C$ (see figure \ref{fig:B13} b). Then $A$ and $B$
can't influence one another and thus won't synchronize.

\begin{figure}
    \centering
    \gnuplotloadfile[terminal=epslatex, terminaloptions={color size 5,2.0}]{pictures/11.gp}
    \caption{The synchronization $r$ of the \enquote{all-to-all} Kuramoto
            model as function of the coupling strength $\sigma$. As
            described in the text, the network size for this an all further
            simulations was chosen to be $N=100$.}
    \label{fig:B11}
\end{figure}

\begin{figure}
    \centering
    \gnuplotloadfile[terminal=epslatex, terminaloptions={color size 6.5,2.5}]{pictures/12.gp}
    \caption{Synchronization of (a) Erdös-Renyi and (b) Barab\'asi-Albert
            networks in the Kuramoto model as function of the coupling
            strength. The results were obtained by averaging over 100 trial
            networks (with varying frequencies $\omega_i$).}
    \label{fig:B12}
\end{figure}

\begin{figure}
    \centering
    \def\svgwidth{0.8\textwidth}
    \input{pictures/13.pdf_tex}
    \caption{Schematic drawing of possible systems that won't synchronize. In
        figure \textbf{(a)} the network is divided in three connected
        components $A$, $B$, and $C$. Figure \textbf{(b)} shows a constellation where the network
        can be split up in three distinct components $A$, $B$, and $C$ such that $A$ and
        $B$ are connected to $C$ only by directed edges going to $C$. In this
        case the oscillations of $A$ and $B$ are completely independend, whereas
        the oscillation of $C$ is influenced by both $A$ and $B$.}
    \label{fig:B13}
\end{figure}


\clearpage

\section{Power Grids}
In order to describe the phase dynamics of \emph{power grids}, models that
are similar to the Kuramoto model from the last section can be used.  In
order to derive these models certain assumptions are made, e.g. that all
consumers and generators work with a similar standard frequency and that
the deviations from this frequency remain small. For so called \emph{smart power
grids}, that try to stabilize the network by local measurements only, one
can furthermore take delayed reactions into account.

One exemplary model for such a smart power grid consisting of two nodes
only -- one generator with phase $\theta_1$ and one consumer with phase
$\theta_2$ -- is given by the following  \emph{delayed differential
equation} (DDE) at time $t$:

\begin{equation}
\begin{aligned}
    \ddot{\theta}_1(t) &= P_1 - \alpha_1\,\dot{\theta}_1(t) + 
                         K\,\sin\big(\theta_2(t) - \theta_1(t)\big) - 
                         \gamma_1\,\dot{\theta}_1(t - \tau)\\
    \ddot{\theta}_2(t) &= P_2 - \alpha_2\,\dot{\theta}_2(t) + 
                         K\,\sin\big(\theta_1(t) - \theta_2(t)\big) - 
                         \gamma_2\,\dot{\theta}_2(t - \tau)~.
\end{aligned}
\end{equation}
Here $P_1, P_2$ are the powers generated/consumed, $\alpha_1, \alpha_2\ge 0$ are
damping constants, $K$ describes the (symmetric) coupling strength, and
$\gamma_1, \gamma_2$ introduce additional delayed damping effects. The
parameter $\tau$ determines the actual time delay of the DDE.

Usually one is only interested in the phase difference $\Delta\theta
\defeq \theta_1 - \theta_2$ as well as $\Delta\omega\defeq \dot{\theta}_1
- \dot{\theta}_2$, since synchronicity of the power grid is already granted
if $\Delta\theta$ stays constant. By assuming that $\alpha \defeq \alpha_1
= \alpha_2$, $\gamma \defeq \gamma_1 = \gamma_2$, and setting $\Delta
P \defeq P_1 - P_2$, one can take the difference of the two above equations and
turn it into two coupled equations of order one:
\begin{equation}\label{eq:smart_grid}
\begin{aligned}
    \Delta\dot{\theta}(t) &= \Delta\omega(t) \\
    \Delta\dot{\omega}(t) &= \Delta P - \alpha\,\omega(t)
    - 2K\,\sin\big(\Delta\theta(t)\big) - \gamma\,\Delta\omega(t-\tau)~.
\end{aligned}
\end{equation}
In the above equations, the delivered power is given by $P\Ix{D}(t)
= K\,\sin\big(\Delta\theta(t))$.

\subsection{Power Grid Oscillators Without Delayed Control}
The first problem to be solved was to simulate the simple power grid
described above without delayed effects (meaning $\gamma = 0$). The
coefficients were taken to be $\alpha = 0.1$, $K = 8$, $P_1 = 1$, and $P_2
= -1$. 

The simulations were carried out using a simple Euler algorithm with
$dt=0.001$ for $0 \le t \le 100$; the results obtained for different
initial conditions can be seen in the figure \ref{fig:B21_1} to \ref{fig:B21_3}.
As long as the initial difference $\Delta\omega_0$ was not chosen to be too
big, the system converged. For great differences however, the system
diverged linearly without carrying out oscillations at all.


\subsection{Simple Delayed Differential Equations}
Before considering similar simulations to the above ones with additional
delayed control, some simple delayed differential equations of the form
\begin{equation}\label{eq:simple_dde}
    \dot{x}(t) = \kappa\,x(t - 1)
\end{equation}
were solved for $\kappa\in\{-1, 1\}$. In figure \ref{fig:B22_1} solutions
for this equation with constant histories are depicted. Plots of $x(t)$
for some other histories (sine waves) are given in figure \ref{fig:B22_2}.
The simulations were carried out using the \texttt{pydelay} python library.

\subsection{Power Grid Oscillators With Delayed Control}
For this problem, simulations of the delayed differential equation
\eqref{eq:smart_grid} were conducted for $\gamma = 0.25$ and $\tau\in\{0, 0.4,
0.8, 1.2, 1.6\}$. The histories were chosen to be $\Delta\theta(t)
= \Delta\omega(t) = 0$ on $t\in[-\tau, 0]$ and the simulation was again
conducted with \texttt{pydelay}. The results are shown in figure
\ref{fig:B23}.

One can see that the stability decreases with increasing $\tau$. While for
$\tau = 1.2$ the convergence is considerably slower than for $\tau = 0.8$,
the networks seems not to converge for $\tau = 1.6$ at all.





\begin{figure}[h]
    \centering
    \gnuplotloadfile[terminal=epslatex, terminaloptions={color size 5,2.5}]{pictures/21_1.gp}
    \caption{Simulation of a power grid consisting of one generator ($P_1
    = 1$) and one consumer ($P_2 = -1$) without delayed control ($\gamma
    = 0$) using the parameters $K=8$ and $\alpha=0.1$. The initial values
    were $\theta_1(0) = \theta_2(0) = 0$,
    $\omega_1(0) = \omega_2(0) = 0$. Figure (a) shows the time development of
    $\theta_1$, $\theta_2$, as well as $\theta_1 - \theta_2$, while (b) shows
    the phase space trajectories $t\mapsto \big(\omega(t), \theta(t)\big)$ for
    both the generator and the consumer.}
    \label{fig:B21_1}
\end{figure}

\begin{figure}[h]
    \centering
    \gnuplotloadfile[terminal=epslatex, terminaloptions={color size 5,2.5}]{pictures/21_2.gp}
    \caption{The same simulation as in figure \ref{fig:B21_1} but with
    initial values $\omega_1(0) = 0.1$ and $\omega_2(0) = -0.05$.}
    \label{fig:B21_2}
\end{figure}

\begin{figure}[h]
    \centering
    \gnuplotloadfile[terminal=epslatex, terminaloptions={color size
    5,2.5}]{pictures/21_3.gp} \caption{The same simulation as in figure
        \ref{fig:B21_1} but with initial values $\omega_1(0) = 5$ and
        $\omega_2(0) = -2.5$. In contrast to figures \ref{fig:B21_1} and
        \ref{fig:B21_2} the system is diverging in this case.}
    \label{fig:B21_3}
\end{figure}

\begin{figure}[h]
    \centering
    \gnuplotloadfile[terminal=epslatex, terminaloptions={color size
    5,2.5}]{pictures/22_const.gp} \caption{Solutions of the delayed
        differential equation
        \eqref{eq:simple_dde} with $\kappa=+1$ in (a) and $\kappa=-1$ in
        (b). The histories are chosen as constants on $t\in[-1, 0]$.
        For $\kappa=+1$ all solutions diverge, while for $\kappa=-1$ all
        solutions converge.}
    \label{fig:B22_1}
\end{figure}

\begin{figure}[h]
    \centering
    \gnuplotloadfile[terminal=epslatex, terminaloptions={color size
    5,2.5}]{pictures/22_sinus.gp} \caption{Solutions of the delayed
        differential equation
        \eqref{eq:simple_dde} with $\kappa=+1$ in (a) and $\kappa=-1$ in
        (b). The histories are chosen as  $x(t) = \sin(c\,\pi\,t)$ with
        $c\in[1, 2.5]$ for $t\in[-1, 0]$. Again, for $\kappa=+1$ all solutions
        diverge, while for $\kappa=-1$ all solutions converge.}
    \label{fig:B22_2}
\end{figure}

\begin{figure}[h]
    \centering
    \gnuplotloadfile[terminal=epslatex, terminaloptions={color size
    5,2.5}]{pictures/23.gp} 
    \caption{Delayed smart grid control simulation of equation
        \eqref{eq:smart_grid} for $\gamma = 0.25$ and $\tau\in \{0, 0.4,
        0.8, 1.2, 1.6\}$. All other parameters were chosen like in the
        simulations done for figure \ref{fig:B21_1}.}
    \label{fig:B23}
\end{figure}

\clearpage

\section{Challenges in Power Grids}
\subsection{Braess' Paradox}
The phenomenon that a network's performance can deteriorate upon the insertion of additional links is known as \emph{Braess' Paradox}. 
In the case of power grids, this means that the network can become unstable (i.e. the critical coupling decreases) when additional power lines are built. 
This can be demonstrated in a grid with three producers and three consumers of equal strength, alternately coupled in a ring topology (fig. \ref{fig:B31svg}). 

Due two the systems symmetry, there are only two qualitatively different possibilities for adding a single new connection:
\begin{enumerate}
 \item connect two sources, or two sinks (dashed lines; this only differs in a sign)
 \item connect a sink with a source diagonally (red, solid lines).
\end{enumerate}

To determine the systems stability, we need to find out how it behaves if it is perturbed around its fixed point. 
We will only consider actual fixed points in the governing equations (i.e. $\dot{\theta} = \omega = 0, \dot{\omega} = 0$), even though the conditions for the electrical grid to operate undisrupted are less strict (e.g. having the same $\dot{\theta}$ at each node would still be stable, albeit describing a net not operating at $50$ Hz).
The fixed point condition is (using cyclic indices)
\begin{align}
 0 = P/K + \sin \left(\theta_i - \theta_{i+1} \right) + \sin \left(\theta_i - \theta_{i-1} \right) \quad \forall i \in \left[ 1, 6\right].
\end{align}
If we look for a solution that shares the symmetry of the system, there are only two different phases: those at producers and those at consumers. Setting one to zero, the other now has to fulfill
\begin{align}
P/(2K) &= \sin \left( \Delta \theta \right)\\
\Delta \theta &= \arcsin \left( \frac{P}{2K}\right) = \arcsin{0.9} = 1.14.
\end{align}
 
We also used this configuration as an initial condition for the modified network, even though it is exacly at the networks fixed point for the second case. In figure \ref{fig:B31}, we show the workload of a single connection of neighbouring nodes. We see that the diagonal connection triggers Braess' paradox. 

Note that our investigation does not guaratee that there is no stable fixed point in the third configuration, because we approximated the initial conditions with those from the symmetric case. However, if one such fixedpoint existed, it either has to be far from the symetric fixed point, or bassin of attraction is vary small. For practical applications this still means that the system is most likely unstable.

\begin{figure}
    \centering
    \def\svgwidth{0.8\textwidth}
    \input{pictures/13.pdf_tex}
    \caption{Schematic drawing of the ring network demonstrating Braess' paradox. 
		Red nodes are consumers and green ones producers.}
    \label{fig:B31svg}
\end{figure}

\begin{figure}[h]
    \centering
    \gnuplotloadfile[terminal=epslatex, terminaloptions={color size
    5,2.5}]{pictures/31.gp} 
    \caption{Relative power transferred by the connection $ 2 \leftrightarrow 3$ for different additional lines (compare figure \ref{fig:B31svg}). For a) and b), the initial perturbation decays, whereas the transmission behaves eratically for c). To make the figure clearer, c) is only shown for every fifth data point. }
    \label{fig:B31}
\end{figure}


\subsection{Interdependent Network Failure}
During the last part of the course, we examined how the failure of nodes affected the functionality of a network. Now, we are considering two networks $A$ and $B$ that mutually depend on one another, 
i.e. each node $a$ in $A$ corresponds to a node $b$ in $B$, and if $a$ no longer operates as intended (either because of failure or because it is no longer connected to the giant component) then $b$ also stops working. 
This can induce a cascade of failure were a single removed node can disable whole areas of the network.

The subset of the network that is still functioning can be determined as the largest mutually connected subgraph. For our simulations, we assume that this is always the intersection between the largest connected subgraphs of $A$ and $B$, which is reasonable as long as both $A$ and $B$ still have a giant component. 

To speed up the computation, we do not calculate different failure probabilities independently, but use one network and sucessivly remove random nodes from it. This means that we have to take care when selecting the node which to remove, because nodes that failed because of cascading failure are still viable options for inherent failure. It also induces some correlation into the data of a single run. 
There are also situations when this algorithm can produce a different result than removing all failing nodes at once: Suppose the network separates into two large components of almost equal size, and the next steps take away nodes from the larger one, such that the other will eventually be larger. If we remove nodes successively, it will be deleted as soon as it is separated, even though at some point it will actually be the largerst mutually connected cluster if we remove nodes simultaniously.
These scenarios, however, become highly unlikely as the network size increases (there is only one giant cluster), so we assume they can be ignored. 

\paragraph{Examples for Interdependent Networks}
We could model an economic system constisting of banks and companies as follows: 
The banks form a network where a link represents credits loaned to another bank, the companies form a network where links denote production chains, i.e. we consider a directed graph. 
Both networks are interdependent in the sense that, if a bank fails, the companies that rely on its credits have to pay them back to the banks creditors and might default too. 
If enough of the companies a bank has lended money to fail and cannot pay back, the bank is bancrupt. 
In this model, the interdependency is no longer a one-to-one mapping as discussed above, because one failing bank might ruin many companies, but in reverse might survive if just a few companies default. 
Also, since the banks loan money to one another, there exist interdependencies within the network of banks itself.

%An example more like the case considered in the lecture an be cosntructed if we split the financial relations into credits loaned and received, i.e. we consider two directed graphs. Then, if 

\end{document}
