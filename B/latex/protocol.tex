\documentclass{scrartcl}

%\usepackage{etex}
\usepackage[left=3cm,right=2.0cm,top=1.25cm,bottom=1.75cm,includehead,includefoot]
{geometry}
\usepackage{marginnote}
\reversemarginpar
%Math
\usepackage{amsfonts,amsmath,amssymb,amsthm, amstext}
%Language, encoding, ...
\usepackage[utf8]{inputenc}
\usepackage[T1]{fontenc}
\usepackage[english]{babel}
\usepackage{t1enc, lmodern, textcomp}
%Pictures
\usepackage{graphicx, color}
%\usepackage{latexsym}
%\usepackage{keyval}
%\usepackage{ifthen}
%\usepackage{moreverb}
\usepackage[shell]{gnuplottex}
%Colors
\usepackage[usenames,dvipsnames]{xcolor}
\definecolor{darkblue}{rgb}{0.0, 0.2, 0.6}
\renewcommand*{\marginfont}{\bf\color{darkblue}}
%Nice a/b fractions
\usepackage{xfrac}
%Make it possible to write (fat) upright greek letters
\usepackage{upgreek}

%For indices
\newcommand{\ix}[1]{_{\mathrm{#1}}}
\newcommand{\Ix}[1]{^{\mathrm{#1}}}

%Directly include svg images
%%%%%%%%%%%%%%%%%%%%%%%%%%%%%%%%%%%%%%%%%%%%%%%%%%
\newcommand{\executeiffilenewer}[3]{%
    \ifnum\pdfstrcmp{\pdffilemoddate{#1}}%
    {\pdffilemoddate{#2}}>0%
    {\immediate\write18{#3}}\fi%
}
\newcommand{\includesvg}[1]{%
    \executeiffilenewer{#1.svg}{#1.pdf}%
    {inkscape -z -D --file=#1.svg %
    --export-pdf=#1.pdf --export-latex}%
    \input{#1.pdf_tex}%
}
%%%%%%%%%%%%%%%%%%%%%%%%%%%%%%%%%%%%%%%%%%%%%%%%%%

%Make captions look okay
\usepackage[font=small,format=plain,labelfont=bf,up,up]{caption}

%Hyperlinks
\usepackage{hyperref}
\hypersetup{
            pdflang=en-EN,
            unicode=true,
            pdfauthor={Thomas Staudt, Erik Schultheis},
}

%Differential d's
\newcommand{\dif}{\mathrm{d}}
\newcommand{\tdif}[2]{\ensuremath{\frac{\dif#1}{\dif#2}}}
\newcommand{\pdif}[2]{\ensuremath{\frac{\partial#1}{\partial#2}}}
\newcommand{\ppdif}[2]{\ensuremath{\frac{\partial^{2}#1}{\partial#2^{2}}}}
%Degree
\newcommand{\degr}{^\circ}

\renewcommand{\refname} {Literature}
\renewcommand{\figurename}{\bf Figure}
\newcommand{\fs}[1]{\footnotesize #1}
%double slash
\newcommand{\git}{\mathbin{
  \mathchoice{\textbackslash\mkern-6mu\textbackslash}% \displaystyle
    {\textbackslash\mkern-6mu\textbackslash}% \textstyle
    {\textbackslash\mkern-5mu\textbackslash}% \scriptstyle
    {\textbackslash\mkern-5mu\textbackslash}}}% \scriptscriptstyle


%Bibliography
\usepackage[babel]{csquotes}
\usepackage[backend=bibtex8]{biblatex}


\newcommand*{\defeq}{\mathrel{\vcenter{\baselineskip0.5ex \lineskiplimit0pt
                     \hbox{\scriptsize.}\hbox{\scriptsize.}}}%
                     =}



%\bibliography{sources}


\begin{document}

\begin{titlepage}\centering
\textsc{\Large Institute For Nonlinear Dynamics \\[1.5ex] Universität Göttingen}

\vspace*{2cm}
{\huge A Practical Course On Network Science}
\vspace*{2cm}

\rule{\textwidth}{1pt}\\[0.5cm]
{\bfseries \huge Block B: \\[0.5cm] \huge \bfseries Oscillating Networks [?]\\[0.5cm]}
\rule{\textwidth}{1pt}

\vspace*{4cm}

\begin{Large}\begin{tabular}{rl}
        \textbf{Participants:}  & Erik Schultheis                                \\    
                   & \textit{erik.schultheis@stud.uni-goettingen.de}\\[0.5cm]
                   & Thomas Staudt                                  \\
                   & \textit{thomas.staudt@stud.uni-goettingen.de}  \\[1.0cm]

       \textbf{Tutors:}        & Dr. Nora Molkenthin, Benjamin Schäfer, Malte Schröder  \\[1.0cm]
       \textbf{Deadline:}      & 09.05.2015
\end{tabular}\end{Large}

\vspace*{1.5cm}


\end{titlepage}

\tableofcontents
\clearpage

\section{Synchronization in Complex Networks}
\subsection{Kuramoto Oscillators}
The \emph{Kuramoto model} describes the phase coupling of a network of
oscillators. A network with $\mathrm{N}$ oscillators that have phases
$(\phi\ix{i}(t))\ix{i=1}\Ix{N}$ at time $t$ and which possess preferential
frequencies $(\omega\ix{i})\ix{i=1}\Ix{N}$ is coupled in the Kuramoto model via the
following system of differential equations:
\begin{equation}
    \label{eq:kuramoto}
    \dot{\phi}\ix{i}(t) = \omega\ix{i} - \sigma~\sum\ix{j=1}\Ix{N} a\ix{ij} \sin\big(\phi\ix{i}(t) - \phi\ix{j}(t)\big)~.
\end{equation}
Here $\sigma$ describes the \emph{coupling strength} and
$(a\ix{ij})\ix{i,j=1}\Ix{N}$ is the adjacency matrix of the unweighted graph of
the network, meaning that $a\ix{ij} \in \{0, 1\}$ for all $1\le
\mathrm{i},\mathrm{j}\le \mathrm{N}$. Besides $\sigma$ the variable $K$
is sometimes used to characterize the coupling strength, where one has the
relation
\begin{equation}
    \sigma = \frac{K}{\mathrm{\langle k \rangle}}
\end{equation}
with the average degree $\langle k \rangle$ of the network.
A possible measure for the synchronization of the oscillators is given by
\begin{equation}
    r(t) = \frac{1}{N}~\left| \sum\ix{j=0}^N \mathrm{e}^{i~\phi\ix{j}(t)} \right|~,
\end{equation}
which is near $0$ for unsynchronized networks and near $1$ for synchronized
ones. The quantity $r$ (averaged over some period of time at the end of
a long simulation) understood as function of the coupling strength
$K$ for a fixed network $a\ix{ij}$ can be used to examine percolation, where
the system suddenly changes from an unsynchronized to a strongly
synchronized state. The critical percolation value $\sigma_c$ for this process is named
the \emph{critical coupling strength}.


\subsection{Critical Coupling for Different Kuramoto Oscillators}
For all of the following simulation results, networks of size $\mathrm{N}=100$ were used.
Single values of $r$ were calculated by numerically solving equation
\eqref{eq:kuramoto} with $\dif t=0.01$ from $t\ix{start} = 0$ up to
$t\ix{end} = 10$ by a simple Euler algorithm, using
values for $\omega_i$ drawn from a standard normal distribution $g \sim
\mathcal{N}(0, 1)$. Then $r$ was taken to be the mean of $r(t)$ over the
last $100$ time steps. 
In order to compare the obtained results with analytic predictions, the
following mean-field approximation was used to calculate $\sigma_c$:
\begin{equation}\label{eq:s_crit}
    \sigma_c = \frac{2}{\pi\,g(0)}~\frac{\langle k \rangle}{\langle k^2 \rangle} = 2\,\sqrt{\frac{2}{\pi}}~\frac{\langle k \rangle}{\langle k^2 \rangle}~.
\end{equation}
Here the second equality holds because of $g \sim \mathcal{N}(0, 1)$.

\paragraph{Completely Connected Networks}
The first simulations were carried out for completely connected networks,
making the coupling \enquote{all-to-all}. In this case, $\omega\ix{ij} = 1$
for all $1\le \mathrm{i},\mathrm{j}\le \mathrm{N}$ and $\langle k \rangle
= \mathrm{N}$ as well as $\langle k^2 \rangle = \mathrm{N}^2$. The results obtained are
presented in figure \ref{fig:B11}. One can see that the analytically
predicted value $\sigma_c = 2/\big(\pi\,g(0)\,\mathrm{N}\big)\approx 0.016$ of
the critical coupling strength is consistent with the results obtained.

\paragraph{Erdös-Renyi Networks}
Next the coupling behaviour of (undirected) Erdös-Renyi networks was
analyzed for three different values of the parameter $p$: $p\in\{0.1, 0.3, 0.5\}$.
Here the predicted values for $\sigma_c$ were calculated numerically by
averaging formula \eqref{eq:s_crit} over 30 trial networks each and are
included in figure \ref{fig:B12} a) together with the simulation results.

One again discerns that the analytic predictions work quite well for
describing the percolation process.  The reason that the needed coupling
strength for synchronization is higher than for the fully connected network
is of course that fewer connections cause weaker mutual influences of the
single oscillators.

\paragraph{Scale Free Networks}
The last network type examined for synchronizing were scale free networks,
implemented by using Barab\'asi-Albert networks with numbers $m\in\{5, 10,
15\}$ of edges added in each step. The critical values were again
calculated numerically as stated in the last paragraph.  The results can be
seen in figure \ref{fig:B12} b), where the analytic values once more seem
to describe the percolation appropriately.

(MISSING: Why are the values the way they are??)

\subsection{Synchronization Conditions}
There are several conditions that are necessary for the network to be able
to synchronize. A very simple requirement is that the network's graph must
be \emph{connected}, since different connected components can't influence the phasing
of one another at all (depicted in figure \ref{fig:B13} a). In general of
course, the stronger the interconnectivity of the oscillators, the better
the synchronization works. 

Another way to destroy the ability to synchronize (besides taking more than
one connected component) is to introduce directed edges and provide some
restrictions for allowed edges. For example, a central component $C$ could
be connected to two other components $A$ and $B$ only by arrows from $A$
and $B$ pointing to $C$ (see figure \ref{fig:B13} b). Then $A$ and $B$
can't influence one another and thus won't synchronize.

\begin{figure}
    \centering
    \gnuplotloadfile[terminal=epslatex, terminaloptions={color size 5,2.0}]{pictures/11.gp}
    \caption{The synchronization $r$ of the \enquote{all-to-all} Kuramoto
            model as function of the coupling strength $\sigma$. As
            described in the text, the network size for this an all further
            simulations was chosen to be $N=100$.}
    \label{fig:B11}
\end{figure}

\begin{figure}
    \centering
    \gnuplotloadfile[terminal=epslatex, terminaloptions={color size 6.5,2.5}]{pictures/12.gp}
    \caption{Synchronization of (a) Erdös-Renyi and (b) Barab\'asi-Albert
            networks in the Kuramoto model as function of the coupling
            strength. The results were obtained by averaging over 100 trial
            networks (with varying frequencies $\omega_i$).}
    \label{fig:B12}
\end{figure}

\begin{figure}
    \centering
    \def\svgwidth{0.8\textwidth}
    \input{pictures/13.pdf_tex}
    \caption{Schematic drawing of possible systems that won't synchronize. In
        figure \textbf{(a)} the network is divided in three connected
        components $A$, $B$, and $C$. Figure \textbf{(b)} shows a constellation where the network
        can be split up in three distinct components $A$, $B$, and $C$ such that $A$ and
        $B$ are connected to $C$ only by directed edges going to $C$. In this
        case the oscillations of $A$ and $B$ are completely independend, whereas
        the oscillation of $C$ is influenced by both $A$ and $B$.}
    \label{fig:B13}
\end{figure}


\clearpage
\section{Power Grids}

\begin{figure}
    \centering
    \gnuplotloadfile[terminal=epslatex, terminaloptions={color size 5,2.0}]{pictures/21.gp}
    \caption{BlahBlah BlahBlahBlah!}
    \label{fig:B21}
\end{figure}


\section{Challenges in Power Grids}
\subsection{Braess' Paradox}
The phenomenon that a networks performance can deteriorate upon the insertion of additional links is known as \empf{Braess' Paradox}. In the case of power grids, it means that the network can become unstable (i.e. the critical coupling decreases) when a power lines are built. This can be demonstrated in a grid with three producers and three consumers of equal strength, alternately coupled in a ring topology (fig. \ref{}). 

Due two the systems symmetry, there are only three qualitatively different possibilities for adding a single new connection:
\begin{enumerate}
 \item connect two sources, or two sinks (this only differs in a sign)
 \item connect a sink with a source diagonally.
\end{enumerate}

To determine the systems stability, we need to find out how it behaves if it is perturbed around its fixed point. We will only consider actual fixed points in the governing equations (i.e. $\dot{\theta} = \omega = 0, \dot{\omega} = 0$), even though the conditions for the electrical grid to operate undisruted are less strict (e.g. having the same $\dot{\theta}$ at each node would still be stable, albeit describing a net not operating at $50$ Hz).
The fixed point condition is (using cyclic indices)
\begin{align}
 0 = P/K + \sin \left(\theta_i - \theta_{i+1} \right) + \sin \left(\theta_i - \theta_{i-1} \right) \quad \forall i \in \left[ 1, 6\right].
\end{align}
If we look for a solution that shares the symmetry of the system, there are only two different phases: those at producers and those at consumers. Setting one to zero, the other now has to fulfill
\begin{align}
P/(2K) &= \sin \left( \Delta \theta \right)\\
\Delta \theta &= \arcsin \left( \frac{P}{2K}\right) = \arcsin{0.9} = 1.14.
\end{align}
 
We also used this configuration as an initial condition for the modified network, even though it is exacly at the networks fixed point for the second case. In figure \ref{}, we show the workload of a single connection of neighbouring nodes. We see that the diagonal connection triggers a Braess' paradox.

 




\end{document}
